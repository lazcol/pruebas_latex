\documentclass[12pt]{article}
\usepackage[spanish]{babel}
\usepackage[latin1]{inputenc}
\usepackage[T1]{fontenc}
\usepackage{amsmath}

% \usepackage{amssymb}
% DATOS PARA EL ENCABEZADO
%\paper{ANÁLISIS MATEMÁTICO III\\Ing. Gabriel Pérez Lance}
\title{TP 02: Funciones en el plano complejo}
%\carrera{INGENIERÍA EN INFORMÁTICA}

\begin{document} 
\let\>=\rangle
\let\<=\langle
\let\pe=\preccurlyeq
\let\minus=\smallsetminus
\let\phi=\varphi
\let\w=\omega
\let\a=\alpha
\let\b=\beta
\def\Z{{\mathbb Z}}
\let\iff=\leftrightarrow
\let\Iff=\Leftrightarrow

%\begin{questions}
%\question Sea $w = f(z) = z(z+1)$, hallar
  \begin{enumerate}
  \item $f(-i)$
  \item $f(2-i)$
  \item $f(3+2i)$
  \item $f(2e^{\frac{3\pi}{4}i})$
  \item $f(e^{-\pi i})$
  \end{enumerate}

%\question Si $w=f(z)=\frac{3z-1}{2z+1}$ válido para $z\neq -\frac{1}{2}$ hallar y graficar
\begin{enumerate}
\item $f(i)$
\item $f(2+i)$
\item $f(2e^{-\frac{\pi}{3}i})$
\item $f(\frac{1}{z})$
\end{enumerate}

%\question Función $w = f(z) = z^2$ 
\begin{enumerate}
\item Definiendo $z=(x+iy)$, $w=(u+iv)$ encontrar las expresiones
\begin{enumerate}
\item $u(x,y)=$
\item $v(x,y)=$
\end{enumerate}
\item Definiendo $z=re^{\alpha i}$, $w=\rho e^{\varphi i}$ encontrar las expresiones
\begin{enumerate}
\item $\rho(r,\alpha)=$
\item $\varphi(r,\alpha)=$
\end{enumerate}
\item Encontrar y graficar el conjunto de complejos $z=(x+iy)$ en el plano $Z$ cuya imagen sea:
\begin{enumerate}
\item $w=(u+iv)$ con $u=c_1 > 0$
\item $w=(u+iv)$ con $u=c_1 < 0$
\item $w=(u+iv)$ con $v=c_2 > 0$
\item $w=(u+iv)$ con $v=c_2 < 0$
\end{enumerate}
\item Encontrar y graficar la imagen de las siguientes regiones definidas en el plano $Z$
\begin{enumerate}
\item $z=re^{i\phi}$ para $0<z<2$ y $ 0<\phi< \frac{\pi}{4}$
\item $z=re^{i\phi}$ para $0<z<2$ y $ \pi<\phi< \frac{5\pi}{4}$
\end{enumerate}
\end{enumerate}

%\question Función exponencial $w = f(z) = e^z$
\begin{enumerate}
\item Si definimos $w = \rho e^{i\phi} = f(z)$, con $z=(x+iy)$ encuentre la expresión de
\begin{enumerate}
\item $\rho = \rho(x,y)$
\item $\phi = \phi(x,y)$
\end{enumerate} 
\item Encontrar los valores de $z$ que verifican las siguientes ecuaciones. Graficar
\begin{enumerate}
\item $e^z=-1$
\item $e^z=-2$
\item $e^z=1+i\sqrt{3}$
\item $e^{2z-1}=1$
\end{enumerate}
\item Encontrar y graficar la imagen de las siguientes regiones definidas en el plano $Z$
\begin{enumerate}
\item $z=(x+iy)$ con $x=c_1 > 0$
\item $z=(x+iy)$ con $y=c_2 < 0$
\end{enumerate}
\item Encontrar y graficar la imagen de la región rectangular definida en el plano Z por las rectas $x=a, x=b$ con $ a<b $, $y=c, y=d$ con $c<d$. 
\end{enumerate}

%\question Función Logarítmica $e^w = z$
\begin{enumerate}
\item Encontrar los valores de $w$ que verifiquen
\begin{enumerate}
\item $e^w=-1-i\sqrt{3}$
\item $e^w=1$
\item $e^w=-ei$
\end{enumerate}
\end{enumerate}

%\question Funciones Trigonométricas
\begin{enumerate}
\item Si $w = (u+iv) = cos (z) = cos(x+iy)$ desarrollar para encontrar las expresiones de $u(x,y), v(x,y)$.\\ Recordar que  $cos(z) = \frac{e^{iz}+e^{-iz}}{2}$.
\item Con las expresiones del punto anterior, verificar que cuando $z = x +i 0$, es decir un número real, se verifica que $w$ también es real y de valor $w=cos(x)$
\item Encontrar la imagen a través de $f(z)=cos(z)$ de una recta horizontal $z=x+ia$. Graficar y analizar
\item Encontrar la imagen a través de $f(z)=cos(z)$ de una recta vertical $z=b+iy$. Graficar y analizar
\item Usando los resultados de los puntos anteriores, encontrar la imagen de una rectangulo definido por las rectas $x=a, x=b, y=c, y=d$ con $a < c < b < d$. Graficar
\item Usando los resultados del punto anterior, encontrar la imagen de un rectangulo definido por las rectas  $x=0, x=\frac{\pi}{2}, y=c, y=d$, con $c<d$. Graficar
\end{enumerate}

%\question Ejercicios adicionales
\begin{enumerate}
\item Utilizando la función compleja $w=f(z)=z^2$ encontrar la región en el plano $Z$ cuya imagen es el cuadrado en el plano $W$ definido por las rectas $u=1, u=2, v=1,$ y $v=2$.
\item Mapear la región en el plano $Z$ definida por $z=re^{i\phi}$ para $0<z<2$ y $ 0<\phi< \frac{\pi}{4}$ utilizando las funciones:
\begin{enumerate}
\item $w = z^3$
\item $w = z^4$
\end{enumerate}
\item Graficar la imagen en el plano $W$ de los siguientes conjuntos, utilizando la función $w=z^2$. Realizar el grafico de ambos conjuntos
\begin{enumerate}
\item la recta de puntos $z=(x+iy)$ con $x=cte > 0$
\item la recta de puntos $z=(x+iy)$ con $y=cte > 0$
\item la recta de puntos $z=(x+iy)$ con $x=y$
\end{enumerate}
\item Realizar los ejercicios del punto 6 para  $sin(z) = \frac{e^{iz}-e^{-iz}}{2}$
\end{enumerate}

%\question Función Inversión $w = f(z) = \frac{1}{z}$
\begin{enumerate}
\item Encontrar el conjunto imagen de las siguientes rectas:
\begin{enumerate}
\item $z=(x+iy)$ con $x=c_1 > 0$ - recta vertical
\item $z=(x+iy)$ con $x=c_1 < 0$ - recta vertical
\item $z=(x+iy)$ con $y=c_2 > 0$ - recta horizontal
\item $z=(x+iy)$ con $y=c_2 < 0$ - recta horizontal
\end{enumerate}
\item Encontrar y graficar en que se transforma el semiplano $A$ defindo por $A=\lbrace z \in \mathbb{C} \therefore 0 < c_1 \leq \mathrm{\Re }(z) \rbrace$
\item Encontrar y graficar la imagen del cuadrante definido por $z \in \mathbb{C} \therefore x > 1 \wedge y > 0$
\end{enumerate}

%\question Función Bilineal $w = \frac{az + b}{cz+ d}$
\begin{enumerate}
\item Encontrar la transformación bilineal que mapea los puntos $z_1 = 2, z_2=i, z_3 = -2$ en los puntos $w_1 = 1, w_2 = i, w_3 = -1$. Graficar
\item Un punto fijo en una transformación $w=f(z)$ es un punto $z_0$ donde se cumple $f(z_0)=z_0$.  Las transformaciones bilineales tienen como máximo 2 puntos fijos.
\begin{itemize}
\item Encontrar los puntos fijos de las transformaciones:
\begin{enumerate}
\item $w=\frac{z-1}{z+1}$
\item $w=\frac{6z-9}{z}$
\end{enumerate}
\item Para la transformación del punto anterior $w=\frac{6z-9}{z}$, encontrar en que se mapea el interior de la región delimitada por el conjunto $A=\lbrace z =(e^{i\phi} + 3), 0 \leq \phi < 2\pi \rbrace$. Graficar
\item Para la transformación del punto anterior $w=\frac{6z-9}{z}$, encontrar en que se mapea el segmento $S= [-i, i]$. Graficar
\end{itemize}
\item Toda transformación bilineal $w = \frac{az + b}{cz+ d}$ puede descomponerse en 4 transformaciones aplicadas en secuencia:
\begin{itemize}
\item $f_1(z) = z + \frac{d}{c}$, traslación en $\frac{d}{c}$
\item $f_2(z) = \frac{1}{z}$, inversión
\item $f_3(z) = \frac{bc-ad}{c^2}z$, homotecia y rotación
\item $f_4(z) = z + \frac{a}{c}$, traslación en $\frac{a}{c}$
\end{itemize}
\begin{enumerate}
\item Descomponer la transformación bilineal $w = \frac{i(1-z)}{1+z}$.
\item Transformar la circunferencia de radio $r=1$ centrada en el origen.\\
En cada paso graficar e indicar las imágenes de los puntos $A=1, B=i, C=-1, D=-i$.
\item Encontrar y graficar la imagen de la región interior a la circunferencia.
\end{enumerate}
\item Graficar la región que se encuentra dentro del disco $\left| z - 2\right| < 2$ y fuera del círculo $\left| z -1 \right| = 1 $
\begin{enumerate}
\item Encontrar la imagen de dicha región cuando es transformada por $w=\frac{-iz+4i}{z}$
\item Graficar, indicando las imágenes de puntos característicos del contorno de la región.
\end{enumerate}
\item Encontrar la transformación bilineal que mapea los puntos  $z_1 = 1, z_2=i, z_3 = -1$ en los puntos $w_1 = 0, w_2 = 1, w_3 = \infty$. Graficar
\begin{itemize}
\item Encuentra y grafica la imagen del interior del triangulo definido por  $z_1, z_2, z_3$
\end{itemize}
\item Encontra la imagen del primer cuadrante $x \geqslant 0, y \geqslant 0$ mediante la transformación $w = \frac{z-1}{z+1}$.
\begin{itemize}
\item Encuentra y grafica la imagen del interior del triangulo definido por  $z_1 = i, z_2 = 1, z_3 = 0$
\end{itemize}
\end{enumerate}

%\end{questions}
\end{document}

