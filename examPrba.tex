\documentclass[11pt]{article}
\usepackage[spanish]{babel}
\usepackage[utf8]{inputenc}
\usepackage[T1]{fontenc}
\usepackage{tikz}
\usepackage{wrapfig}
\usepackage{xfrac}

\usepackage[matrix,arrow]{xy}
\usetikzlibrary{intersections}

\usepackage{subcaption}
\usepackage{exam,amssymb}

\DeclareMathSizes{12}{12}{10}{8}

% DATOS PARA EL ENCABEZADO
\paper{ANÁLISIS MATEMÁTICO III}
\title{Exámen - 17 de Febrero de 2021}
\carrera{INGENIERÍA EN INFORMÁTICA}

\begin{document} 
\let\>=\rangle
\let\<=\langle
\let\pe=\preccurlyeq
\let\minus=\smallsetminus
\let\phi=\varphi
\let\w=\omega
\let\a=\alpha
\let\b=\beta
\def\Z{{\mathbb Z}}
\let\iff=\leftrightarrow
\let\Iff=\Leftrightarrow


%\framebox[\textwidth][l]{Apellido, Nombre:} \par
\textbf{IMPORTANTE:} La justificación es fundamental y debe estar presente en cada ejercicio con rigor matemático.


\begin{questions}
\question Determinar y graficar la imagen de la función
$$f(z)=2 e^{z}+1+i$$
 con $z$ que es el segmento que va desde $0$ hasta $4i$, cuando es transformada mediante la función
 $$g(z)=\frac{z(1+i)+2}{2z+(1-i)}$$
 Determinar claramente la ubicación de la imagen de los siguientes puntos $A=(0 + 1i)$ y $B= (0+3i)$.
%\begin{figure}[h!]
%	\begin{tikzpicture}
%	\draw[->, gray] (0,-1)--(0,3);
%	\draw[->, gray] (-3,0)--(1,0);
%	\draw[-, very thick] (0,2)  arc[radius = 1.41, start angle= 45, end angle= 225];
%	\draw[-, gray] (-2,0)  arc[radius = 1.41, start angle= 225, end angle= 405];
%	\node at (0.25, 2.1){2i};
%	\node at (-2.25, -0.25){-2};
%	\end{tikzpicture}
%	\caption{ }
%	\label{fig:ejercicio1}
%\end{figure}


\question Calcular (si existe)
$$ \lim_{z \to i} \frac{z\,\Re(iz)+ i}{i-z} $$
En caso de no existir, justificar {\bfseries{ por qué no existe}}, y de ser posible dar un contraejemplo.

\question Determinar la derivabilidad de la función
$$f(z)= \mathrm{\Re }(\overline{z})^2 +i \mathrm{\Im }(z)^2$$
  
%\question Calcular la siguiente integral, por el camino indicado en la figura \ref{fig:subfig4}. $$\int_{\gamma} \frac{i(z-1)^2}{3(z^2-1)}\,\mathrm{d}z$$
%\begin{figure}[h!]
%\centering
%	\begin{tikzpicture}
%		\draw[->, gray] (0,1)--(0,-3);
%		\draw[->, gray] (-3,0)--(3,0);
%		\node at (3, -0.25) {x}; 
%		\node at (0.35, -3) {-y}; 
%		\draw[->, very thick] (-2,0)  arc[radius = 2, start angle= 180, end angle= 360];
%		\node at (1.35, -1.9) {$\gamma$};
%		\node at (2,0.25) {2};
%		\node at (-2,0.25) {-2};
%		\end{tikzpicture}
%		\caption{}
%        \label{fig:subfig4}
%\end{figure}

\question Calcular, por el camino indicado en la figura \ref{fig:subfig5}, la siguiente integral:
$$\oint_{C} \frac{z-2}{z^2-2z+2}\,\mathrm{d}z$$

\begin{figure}[h!]
\centering
	\begin{tikzpicture}[scale=0.5]
		\draw[->, gray] (0,-1)--(0,4);
		\draw[->, gray] (-4,0)--(4,0);
		\node at (4.4, -0.25) {x}; 
		\node at (0.25, -4) {y}; 
		\draw[->, very thick] (0,-3)  arc[radius = 3, start angle= -90, end angle= 90];
		\draw[->, very thick] (0,3) -- (0,-3);
		\node at (3.25,-0.25) {3};
		\node at (-0.5,3.15) {3i};
		\node at (-0.5,-3.35) {-3i};
		\node at (2.15, -2.7) {$C$};
		\end{tikzpicture}
		\caption{}
        \label{fig:subfig5}
\end{figure}

\question Encontrar el desarrollo en serie de Laurent centrado en $z=-3$ de la función
$$w(z) = \frac{z+i}{z^2+2\ z-8}$$


\end{questions}
\end{document}



\question
El sistema de codificación por tonos (DTMF) usado por el sistema telefónico, codifica sumando señales $f(t) = \cos
(f_0 t)$. En particular un dígito del teclado es codificado sumando dos señales cosenoidales de distinta frecuencia:
\begin{center}
$D_x(t) = \cos (f_A t) + \cos (f_B t)$
\end{center}
Por ejemplo para el dígito 3, de acuerdo a la tabla:
\begin{center}
$D_3(t) = \cos (697 t) + \cos (1477 t)$
\end{center}
\begin{enumerate}
\item Encontrar la transformada de Fourier para la señal de ejemplo $D_3(t)$.
\item Dibujar el espectro de la señal.
\item La señal es digitalizada utilizando una frecuencia de muestreo de 10KHz. Dibujar el espectro de la señal digitalizada.
\item Si la máxima frecuencia a transmitir por la línea telefónica es de 4Hz, ¿Cuál es la frecuencia mínima de muestreo que puede emplearse para digitalizar las señales?.
\end{enumerate}

\end{questions}
\end{document}

